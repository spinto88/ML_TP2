\documentclass[11pt, spanish]{article}
\usepackage[spanish]{babel}
\selectlanguage{spanish}
\usepackage[utf8]{inputenc}
\usepackage{graphicx}
\usepackage{mathtools}
\usepackage{tabularx}
\usepackage[font=small,labelfont=bf]{caption}
\usepackage{subcaption}
\usepackage{authblk}

\captionsetup[table]{name=Tabla}
\renewcommand{\thetable}{\Roman{figure}}
\newcommand{\mean}[1]{\left\langle#1\right\rangle}
%\newcommand{\eqref}[1]{Ec.~\ref{#1}}

\title{TP: Q-learning}

\begin{document}

\maketitle

\section{Introducción}
\par En este trabajo exploramos el algoritmo de Q-learning para entrenar 
jugadores virtuales del juego \emph{4 en línea}. El algoritmo, en su versión 
más básica, consiste en explorar la mayor cantidad de configuraciones 
posibles del juego, penalizando o recompensando las distintas acciones 
que los jugadores realizan. Al finalizar el 
entrenamiento los jugadores virtuales tienen una valorización de algunas de las 
acciones posibles para una fracción del total de configuraciones posibles del 
juego. 

\subsection{\emph{4 en línea}}

\par El \emph{4 en línea} es un juego de mesa para dos jugadores que consiste 
en introducir fichas en un tablero vertical, con el objetivo de colocar cuatro 
fichas consecutivas del color correspondiente, ya sea en forma vertical, 
horizontal, o diagonal. Gana el primer jugador que alcanza esa configuración, y 
en caso que el tablero se complete antes de que algún jugador lo logre, se 
produce un empate. 
Típicamente el tablero tiene un tamaño de 6 filas por 7 columnas, lo que da un 
total de 4.531.985.219.092 configuraciones posibles.

\section{Algoritmo de Q-learning}

\par El algoritmo consiste en explorar la mayor cantidad de configuraciones 
posibles del juego. Durante el entrenamiento (etapa de exploración), los 
jugadores actualizan una magnitud $Q(s,a)$ que representa la ganancia o 
valorización de realizar la acción $a$ en el estado $s$. Cuando un jugador 
realiza una acción ganadora (aquella que lo conduce a un estado con cuatro 
fichas alineadas) recibe una recompensa lo que lo lleva, en el 
transcurso del entrenamiento, a valorizar positivamente las 
acciones que lo acercaron a dicho estado. 
\par Finalizado el entrenamiento, la mejor estrategia teórica consiste en, dado 
un estado $s$, elegir la acción $a$ que maximize $Q(s,a)$ (etapa de 
explotación).

\subsection{Estructura de datos}

\par Almacenamos la ganancia $Q(s,a)$ en un diccionario cuyas entradas son un 
estado representado con una variable \emph{string} de 42 caracteres de largo. 
Así por ejemplo el \emph{string} '0021000....0' representa un estado donde 
el primer jugador tiene una ficha en la cuarta columna, el segundo jugador en 
la tercera, y el resto del tablero se encuentra vacío. Inicializamos los valores 
de $Q(s,a)$ con un número real aleatorio entre 0 y 1.

\subsection{Representación del problema}

\par Decidimos pensar el juego como la evolución de un único sistema que puede 
encontrarse en cualquiera de los estados mencionados en la introducción. Para 
ello en lugar de distinguir entre dos jugadores observamos que para cada estado 
queda inmediatamente determinado cuál de ambos contrincantes debe mover por la 
paridad del número de fichas del mismo; así si el sistema tiene una cantidad par 
de fichas, necesariamente juega el primer jugador en alguna de las columnas 
disponibles, caso contrario lo hace el segundo.
\par Por otro lado, consideramos que una correcta actualización de la ganancia 
$Q(s,a)$ viene dada por la ecuación \ref{eq:Q}, donde $R(s,a)$ es la recompensa 
de realizar una acción $a$ en el estado $s$, $s'$ es el estado obtenido al 
realizar $a$ en $s$, y $a'$ es cualquiera de la acciones posibles en el estado 
$s'$.
\begin{equation}
Q^{n+1}(s,a) = Q^{n}(s,a) + \alpha (R(s,a) - \gamma (max_{a'}Q^n(s',a') - Q^n(s,a))
\label{eq:Q}
\end{equation} 
Dado un estado, la valorización de una acción:
\begin{itemize}
\item aumenta si al realizar una acción, de esta se obtiene una recompensa (en este caso se gana el juego)
\item disminuye si la mejor acción del siguiente estado (que corresponde al otro jugador) está muy valorizada. Es decir cada jugador trata de tomar acciones que no lleven a un estado donde el otro jugador tenga una muy buena jugada.
\end{itemize}
Pensándolo de esta manera, en el algoritmo solo se introducen recompensas 
positivas. Cuando un jugador gane y reciba una recompensa, el otro jugador 
desvaluará la acción que llevó al sistema a ese estado, y valorará más las 
otras.

\subsection{Exploración de los estados: temperatura}

\par Durante la fase de entrenamiento no existe \emph{a priori} ninguna 
restricción sobre las acciones que el sistema decida explorar; estas pueden ser 
escogidas totalmente al azar. Siguiendo esta estrategia, y al cabo de 
cierto tiempo, la exploración recorrerá múltiples veces todos los pares 
$(s,a)$ logrando una valorización adecuada de todas las acciones posibles para 
todos los estados. Sin embargo dada la ingente cantidad de estados posibles 
mencionada en la introducción, aun en un tablero pequeño esta estrategia 
resulta impracticable en un tiempo razonable. Resulta entonces útil introducir 
una magnitud (temperatura) que permita focalizar la búsqueda en aquellas ramas 
del árbol de jugadas que, a la luz del entrenamiento transcurrido, parezcan la 
más promisorias. Cuando la temperatura sea máxima el sistema elegirá en forma 
equiprobable entre las acciones posibles (exploración pura), mientras que 
cuando sea mínima escogerá aquella acción que maximize $Q$ (explotación pura).
\par En suma, la disminución progresiva de este parámetro nos permite variar la 
estrategia de entrenamiento desde una puramente exploratoria y abarcativa hacia 
otra focalizada sobre las ramas más promisorias del árbol de jugadas.

\subsection{Modificaciones agregadas}
ACA IRIAN LAS COSAS QUE LE PUSIMOS DESPUES, DE QUE GANE EL JUEGO SI TIENE 3, ETC...

\section{Resultados}

\section{Conclusiones}

\end{document}
